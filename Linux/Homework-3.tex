% Koko
\documentclass[blue,normal,cn]{elegantnote}
\usepackage{array}
\usepackage{courier}
\usepackage{xcolor}
\definecolor{light-gray}{gray}{0.95}
\newcommand{\code}[1]{\colorbox{light-gray}{\texttt{#1}}}
\newfontfamily\courier{Courier New}
\lstset{linewidth=1.1\textwidth,
	numbers=left,
	basicstyle=\small\courier,
	numberstyle=\tiny\courier,
	keywordstyle=\color{blue}\courier,
	commentstyle=\it\color[cmyk]{1,0,1,0}\courier, 
	stringstyle=\it\color[RGB]{128,0,0}\courier,
	frame=single,
	backgroundcolor=\color[RGB]{245,245,244},
	breaklines,
	extendedchars=false, 
	xleftmargin=2em,xrightmargin=2em, aboveskip=1em,
	tabsize=4, 
	showspaces=false
	basicstyle=\small\courier
}
\title{上机作业3:Shell脚本程序设计}
\version{$\aleph$}
\date{\today}

\begin{document}
\author{
	\begin{tabular}[t]{c}
		088于海鑫  \\
		2017211240 \\
		\emph{name1e5s@bupt.edu.cn}
	\end{tabular}
}
\maketitle

\section{第一题:生成TCP活动状况报告}

\subsection{题目要求}

\code{netstat --statistics} 命令可以列出 \code{tcp} 等协议的统计信息。编写 \code{shell} 脚本程序,每隔 1 分钟生成 1 行信息:当前时间;这一分钟内 TCP 发送了多少报文;接收了多少报文;收发报文总数;行尾给出符号+或-或空格(\code{+}表示这分钟收发报文数比上分钟多10包以上,差别在10包或以内用空格,否则用符号\code{-})。

\subsection{解法}

\subsubsection{获取信息}

执行 \code{netstat --statistics},获得信息如下:

\begin{lstlisting}
... 
Tcp:
    3634892 active connection openings
    4228 passive connection openings
    3506040 failed connection attempts
    5947 connection resets received
    15 connections established
    9911704 segments received
    10086150 segments sent out
    13700 segments retransmitted
    8 bad segments received
    3514876 resets sent
...
\end{lstlisting}

我们需要的就是 \code{9911704 segments received} 以及
\code{10086150 segments sent out} 这两部分的数据。

使用 \code{直接获取之}。

\begin{lstlisting}
$ COMMAND=$(netstat --statistics)
$ RECV=$(expr "$COMMAND" : '^.* \([0-9]*\) segments received')
$ SENT=$(expr "$COMMAND" : '^.* \([0-9]*\) segments sent out')
$ echo $RECV $SENT
9914989 10090134
\end{lstlisting}

\subsection{编写完整程序}
在上一节的基础上,我们加上一些循环的代码,即可得到结果。

\begin{lstlisting}[language=bash]
#!/bin/bash

FIRST_FLAG=1

COMMAND=$(netstat --statistics)
RECV_PREV=$(expr "$COMMAND" : '^.* \([0-9]*\) segments received')
SENT_PREV=$(expr "$COMMAND" : '^.* \([0-9]*\) segments sent out')
TOTAL_PREV=0

while true
do
        sleep 60
        DATE=$(date +"%Y-%m-%d %H:%M")
        COMMAND=$(netstat --statistics)
        RECV=$(expr "$COMMAND" : '^.* \([0-9]*\) segments received')
        SENT=$(expr "$COMMAND" : '^.* \([0-9]*\) segments sent out')
        R_DELTA=$(expr $RECV - $RECV_PREV)
        S_DELTA=$(expr $SENT - $SENT_PREV)
        TOTAL=$(expr $R_DELTA + $S_DELTA)
        TOTAL_DELTA=$(expr $TOTAL - $TOTAL_PREV)

        if [ $FIRST_FLAG = 1 ]
        then
                echo $DATE $S_DELTA $R_DELTA $TOTAL
                FIRST_FLAG=0
        else
                if [ $TOTAL_DELTA -gt 10 ]
                then
                        echo $DATE $S_DELTA $R_DELTA $TOTAL +
                elif [ $TOTAL_DELTA -lt 0 ]
                then
                        echo $DATE $S_DELTA $R_DELTA $TOTAL -
                else
                        echo $DATE $S_DELTA $R_DELTA $TOTAL
                fi
        fi
        RECV_PREV=$RECV
        SENT_PREV=$SENT
        TOTAL_PREV=$TOTAL
done
\end{lstlisting}

结果如下:

\begin{lstlisting}
name1e5s@sumeru:~$ bash test.sh
2020-05-01 19:46 365 253 618
2020-05-01 19:47 335 239 574 -
2020-05-01 19:48 362 246 608 +
2020-05-01 19:49 373 250 623 +
2020-05-01 19:50 350 245 595 -
2020-05-01 19:51 376 260 636 +
2020-05-01 19:52 400 275 675 +
2020-05-01 19:53 368 250 618 -
2020-05-01 19:54 389 265 654 +
2020-05-01 19:55 391 274 665 +
2020-05-01 19:56 416 279 695 +
2020-05-01 19:57 380 259 639 -
2020-05-01 19:58 451 293 744 +
2020-05-01 19:59 381 267 648 -
2020-05-01 20:00 411 280 691 +
2020-05-01 20:01 427 291 718 +
2020-05-01 20:02 399 270 669 -
\end{lstlisting}

符合要求。

\section{第二题:下载bing图库中图片}

\subsection{题目要求}

访问 \code{https://bing.ioliu.cn/?p=23} 可以看到 bing 图库第23页的内容,这个Web页有多个图片小样,将鼠标放到某个小样上,如右上角,可见中文说明信息 “野花草甸上的一只欧亚雕鸮,德国莱茵兰-普法尔茨”和日期信息2019-08-03,点击一下,
此图片就可以下载。

编写脚本程序bing.sh,将这个图库中照片全部下载下来存放到本地bing目录中,上面URL中
p=23可以换成p=1到p=126可访问126个页,每页有12个图,每个图的日期,中文说明信息和
下载地址及文件名html文件中可提取。要求下载后的文件命名为“日期 中文说明.jpg”

例如:
2019-08-03 野花草甸上的一只欧亚雕鸮,德国莱茵兰-普法尔茨.jpg

\subsection{获取结果}
首先使用 Chrome 的调试功能,确定出来单个小样的表示形式如下:

\begin{lstlisting}[language=HTML]
<div class="item">
  <div class="card progressive">
	<img class="progressive__img progressive--is-loaded" 
	data-progressive="http://h1.ioliu.cn/bing/BurgAltdahn_ZH-CN8281669977_640x480.jpg?imageslim" 
	src="http://h1.ioliu.cn/bing/BurgAltdahn_ZH-CN8281669977_640x480.jpg?imageslim">
    <a class="mark" href="/photo/BurgAltdahn_ZH-CN8281669977?force=home_1"></a>
    <div class="description">
      <h3>达恩附近普法尔茨森林中的Altdahn城堡,德国莱茵兰-普法尔茨(Dahn Rockland), Palatinate Forest, Rhineland-Palatinate, Germany (© Reinhard Schmid/Huber/eStock Photo)</h3>
      <p class="calendar">
        <i class="icon icon-calendar"></i>
        <em class="t">2020-05-01</em></p>
      <p class="view">
        <i class="icon icon-eye"></i>
        <em class="t">453</em></p>
    </div>
	<div class="options">
	... 
    </div>
  </div>
</div>
\end{lstlisting}

显然我们需要的只有 \code{mark} 下面的 \code{href} 以及 \code{description} 下面的描述。

显然我们使用如下指令即可匹配需要的数据:

\begin{lstlisting}[language=bash]
cat index.html | sed -e 's/<div class="item">/\n/g' | \
grep "<div class=\"card progressive\"" | \
sed -e 's#.*\"mark\" href=\"\(.*\)\".*\"description\"><h3>\(.*\)(©.*\([0-9]\{4\}\-[0-9]\{2\}-[0-9]\{2\}\)<.*#\1$\2$\3#g' \
	-e 's#home_1#download#g' \
	-e 's# $2#$2#g'
\end{lstlisting}

输出如下:

\begin{lstlisting}
/photo/BurgAltdahn_ZH-CN8281669977?force=download$达恩附近普法尔茨森林中的Altdahn城堡,德国莱茵兰-普法尔茨(Dahn Rockland), Palatinate Forest, Rhineland-Palatinate, Germany$2020-05-01
/photo/ArcticRedpoll_ZH-CN7968973967?force=download$极北朱顶雀的巢,芬兰拉普兰$2020-04-30
/photo/PalouseSpring_ZH-CN6803103328?force=download$普尔曼附近的帕卢斯一辆拖拉机在耕作时扬起尘土,华盛顿州$2020-04-29
/photo/SalisburyCathedral_ZH-CN6366350896?force=download$索尔茲伯里大教堂与放牧的羊群,英格兰$2020-04-28
/photo/SouthAmericanTapir_ZH-CN6151058361?force=download$一只南美貘幼崽小跑着穿过草地$2020-04-27
/photo/RubySunset_ZH-CN5544596519?force=download$红宝石海滩的日落,华盛顿州奥林匹克国家公园$2020-04-26
/photo/FalklandRockhoppers_ZH-CN5370686595?force=download$福克兰群岛上的南跳岩企鹅$2020-04-25
/photo/MegellanicCloud_ZH-CN5132305226?force=download$由哈勃太空望远镜拍摄的大麦哲伦星云$2020-04-24
/photo/KingEider_ZH-CN3559595357?force=download$正在游水的雄性王绒鸭,挪威特罗姆斯-芬马克郡$2020-04-23
/photo/KauriTree_ZH-CN3695568740?force=download$怀波瓦森林中一棵名为Te Matua Ngahere的巨型贝壳杉树 ,新西兰北地$2020-04-22
/photo/GPS_ZH-CN5160095061?force=download$黄石国家公园的大棱镜泉,怀俄明州$2020-04-21
/photo/BluebellWood_ZH-CN8128422960?force=download$Micheldever Wood的蓝铃花,英国汉普郡$2020-04-20
\end{lstlisting}

之后就是简单的多线程下载等步骤,全部代码如下:

\begin{lstlisting}[language=bash]
if [ $# -gt 2 ]
then
	echo "Usage: $0 <begin> <end>"
	exit 1
fi

if [ $# = 0 ]
then
	START=1
	END=126
else
	START=$1
	END=$2
fi

if [ ! -d bing ]
then 
	mkdir bing
fi

if [ ! -d html ]
then 
	mkdir html
fi

if [ ! -d temp ]
then 
	mkdir temp
fi

for i in $(seq $START $END)
do
	if [ ! -f html/$i.html ]
	then
		wget --no-verbose -O html/$i.html https://bing.ioliu.cn/?p=$i
	fi

	if [ ! -f temp/$i.dat ]
	then
		cat html/$i.html | \
		sed -e 's/<div class="item">/\n/g' | \
		grep "<div class=\"card progressive\"" | \
		sed -e 's#.*\"mark\" href=\"\(.*\)\".*\"description\"><h3>\(.*\)(©.*\([0-9]\{4\}\-[0-9]\{2\}-[0-9]\{2\}\)<.*#\1$\2$\3#g' \
		 -e 's#home_1#download#g' \
		 -e 's# $2#$2#g' | \
		awk -F "[$]" '{ printf("https://bing.ioliu.cn/%s$%s$%s\n", $1, $2, $3)}' > temp/$i.dat
	fi
done

for i in $(seq $START $END)
do
	while read line
	do
 		eval $(echo "$line" | awk -F "[$]" '{printf("url=%s;name=\"%s\";date=%s",$1,$2,$3)}')
		img="bing/$date $name.jpg"
		if [ ! -f "$img" ]
 		then
			tmp="$img.tmp$$"
			wget --no-verbose -O "$tmp" $url
			if [ $? = 0 -a ! -f "$img" ]
			then
				mv "$tmp" "$img"
			else
				rm "$tmp"
			fi
		fi
	done < temp/$i.dat
done
\end{lstlisting}

运行结果如下(部分):

\begin{lstlisting}
name1e5s@sumeru:~/t$ bash a.sh 1 1
2020-05-01 22:21:29 URL:https://bing.ioliu.cn/?p=1 [23648/23648] -> "html/1.html" [1]
2020-05-01 22:21:33 URL:https://bing.ioliu.cn//photo/BurgAltdahn_ZH-CN8281669977?force=download [290069] -> "bing/2020-05-01 达恩附近普法尔茨森林中的Altdahn城堡,德国莱茵兰-普法尔茨(Dahn Rockland), Palatinate Forest, Rhineland-Palatinate, Germany.jpg.tmp5340" [1]
2020-05-01 22:21:40 URL:https://bing.ioliu.cn//photo/ArcticRedpoll_ZH-CN7968973967?force=download [327709] -> "bing/2020-04-30 极北朱顶雀的巢,芬兰拉普兰.jpg.tmp5340" [1]
2020-05-01 22:21:44 URL:https://bing.ioliu.cn//photo/PalouseSpring_ZH-CN6803103328?force=download [181230] -> "bing/2020-04-29 普尔曼附近的帕卢斯一辆拖拉机在耕作时扬起尘土,华盛顿州.jpg.tmp5340" [1]
2020-05-01 22:21:49 URL:https://bing.ioliu.cn//photo/SalisburyCathedral_ZH-CN6366350896?force=download [284849] -> "bing/2020-04-28 索尔茲伯里大教堂与放牧的羊群,英格兰.jpg.tmp5340" [1]
2020-05-01 22:21:52 URL:https://bing.ioliu.cn//photo/SouthAmericanTapir_ZH-CN6151058361?force=download [297595] -> "bing/2020-04-27 一只南美貘幼崽小跑着穿过草地.jpg.tmp5340" [1]
2020-05-01 22:21:55 URL:https://bing.ioliu.cn//photo/RubySunset_ZH-CN5544596519?force=download [310129] -> "bing/2020-04-26 红宝石海滩的日落,华盛顿州奥林匹克国家公园.jpg.tmp5340" [1]
2020-05-01 22:21:57 URL:https://bing.ioliu.cn//photo/FalklandRockhoppers_ZH-CN5370686595?force=download [142025] -> "bing/2020-04-25 福克兰群岛上的南跳岩企鹅.jpg.tmp5340" [1]
2020-05-01 22:22:01 URL:https://bing.ioliu.cn//photo/MegellanicCloud_ZH-CN5132305226?force=download [338962] -> "bing/2020-04-24 由哈勃太空望远镜拍摄的大麦哲伦星云.jpg.tmp5340" [1]
2020-05-01 22:22:03 URL:https://bing.ioliu.cn//photo/KingEider_ZH-CN3559595357?force=download [140302] -> "bing/2020-04-23 正在游水的雄性王绒鸭,挪威特罗姆斯-芬马克郡.jpg.tmp5340" [1]
2020-05-01 22:22:05 URL:https://bing.ioliu.cn//photo/KauriTree_ZH-CN3695568740?force=download [343549] -> "bing/2020-04-22 怀波瓦森林中一棵名为Te Matua Ngahere的巨型贝壳杉树 ,新西兰北地.jpg.tmp5340" [1]
2020-05-01 22:22:08 URL:https://bing.ioliu.cn//photo/GPS_ZH-CN5160095061?force=download [292443] -> "bing/2020-04-21 黄石国家公园的大棱镜泉,怀俄明州.jpg.tmp5340" [1]
2020-05-01 22:22:10 URL:https://bing.ioliu.cn//photo/BluebellWood_ZH-CN8128422960?force=download [269926] -> "bing/2020-04-20 Micheldever Wood的蓝铃花,英国汉普郡.jpg.tmp5340" [1]
\end{lstlisting}

完成后打开 \code{bing} 文件夹,即可看到文件。

\begin{lstlisting}
name1e5s@sumeru:~/t/bing$ ls
'2020-04-20 Micheldever Wood的蓝铃花,英国汉普郡.jpg'
'2020-04-21 黄石国家公园的大棱镜泉,怀俄明州.jpg'
'2020-04-22 怀波瓦森林中一棵名为Te Matua Ngahere的巨型贝壳杉树 ,新西兰北地.jpg'
'2020-04-23 正在游水的雄性王绒鸭,挪威特罗姆斯-芬马克郡.jpg'
'2020-04-24 由哈勃太空望远镜拍摄的大麦哲伦星云.jpg'
'2020-04-25 福克兰群岛上的南跳岩企鹅.jpg'
'2020-04-26 红宝石海滩的日落,华盛顿州奥林匹克国家公园.jpg'
'2020-04-27 一只南美貘幼崽小跑着穿过草地.jpg'
'2020-04-28 索尔茲伯里大教堂与放牧的羊群,英格兰.jpg'
'2020-04-29 普尔曼附近的帕卢斯一辆拖拉机在耕作时扬起尘土,华盛顿州.jpg'
'2020-04-30 极北朱顶雀的巢,芬兰拉普兰.jpg'
'2020-05-01 达恩附近普法尔茨森林中的Altdahn城堡,德国莱茵兰-普法尔茨(Dahn Rockland), Palatinate Forest, Rhineland-Palatinate, Germany.jpg'
\end{lstlisting}

\section{作业总结}

本次上机作业比较耗费时间,主要是要回顾之前学过的内容。

通过本次作业,我对 \code{sed} 的使用有了更加深刻的了解。

\end{document}