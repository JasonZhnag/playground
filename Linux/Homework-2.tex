% Koko
\documentclass[blue,normal,cn]{elegantnote}
\usepackage{array}
\usepackage{courier}
\usepackage{xcolor}
\definecolor{light-gray}{gray}{0.95}
\newcommand{\code}[1]{\colorbox{light-gray}{\texttt{#1}}}
\newfontfamily\courier{Courier New}
\lstset{linewidth=1.1\textwidth,
	numbers=left,
	basicstyle=\small\courier,
	numberstyle=\tiny\courier,
	keywordstyle=\color{blue}\courier,
	commentstyle=\it\color[cmyk]{1,0,1,0}\courier, 
	stringstyle=\it\color[RGB]{128,0,0}\courier,
	frame=single,
	backgroundcolor=\color[RGB]{245,245,244},
	breaklines,
	extendedchars=false, 
	xleftmargin=2em,xrightmargin=2em, aboveskip=1em,
	tabsize=4, 
	showspaces=false
	basicstyle=\small\courier
}
\title{上机作业2:遍历目录}
\version{$\aleph$}
\date{\today}

\begin{document}
\author{
	\begin{tabular}[t]{c}
		088于海鑫  \\
		2017211240 \\
		\emph{name1e5s@bupt.edu.cn}
	\end{tabular}
}
\maketitle

\tableofcontents

\section{作业要求}

编程实现程序 \code{list.c},列表普通磁盘文件,包括文件名和文件大小。

\begin{enumerate}
	\item 使用 \code{vi} 编辑文件,熟悉工具 \code{vi}
	\item 使用 Linux 的系统调用和库函数
	\item 体会命令对选项的处理方式
\end{enumerate}

对选项的处理,自行编程逐个分析命令行参数。不考虑多选项挤在一个命令行
参数内的情况。

\section{作业实现}

\subsection{获取选项}

因为在这次作业中,使用 \code{getopt} 是不被允许的,因此我们
首先需要一个用于处理选项的函数。在我们的实现中,我们决定基于
C 标准库函数 \code{strchr} 模拟实现一个 \code{getopt} 以
支持我们的选项分析。核心代码如下:

\begin{lstlisting}[language=C]
    optdecl = strchr(optstring, optchar);
    if (optdecl) {
        if (optdecl[1] == ':') {
            optarg = ++optcursor;
            if (*optarg == '\0') {
                if (++optind < argc) {
                    optarg = argv[optind];
                } else {
                    optarg = NULL;
                    optchar = (optstring[0] == ':') ? ':' : '?';
                }
            }
            optcursor = NULL;
        }
    } else {
        optchar = '?';
    }
\end{lstlisting}

\subsection{输出单个文件}
在输出单个文件时,我们使用 \code{stat} 函数来获取单个文件的状态信息,之后根据这些状态信息打印出文件。核心代码如下:

\begin{lstlisting}[language=C]
static void list_node(const char *prefix, const char *file_name) {
	// 一些拼接代码

    struct stat status;
    if(stat(node_name, &status)) {
        fprintf(stderr, "%s - Can't access \"%s\": %s\n", elf_name, node_name, strerror(errno));
        return;
    }

    mode_t stat_mode = status.st_mode;
    off_t stat_size = status.st_size;
	time_t stat_mtime = status.st_mtime;
	
    // 一些格式化代码...

    int filter_flag = (!modify_time || time (NULL) - stat_mtime <= modify_time) &&
        (!lo_size || stat_size >= lo_size) &&
        (!hi_size || stat_size <= hi_size);
    if((filter_flag && !S_ISDIR(stat_mode)) || (S_ISDIR(stat_mode) && recursive_flag)) {
        char stat_time_str[64];
        strftime(stat_time_str, 64, 
            "%Y-%m-%d %H:%M", localtime(&stat_mtime));
        printf ("%s %8ld %s %s\n", mode_text, stat_size, stat_time_str, file_name);
    }
	
	// 一些收尾代码
}
\end{lstlisting}

\subsection{输出文件夹下的全部文件}

基于上面输出单个文件的代码,我们可以实现输出某文件夹下全部文件的
函数。其核心在于对 \code{readdir} 函数的使用。代码如下:

\begin{lstlisting}[language=C]
static void list_dir(const char *name) {
    if(!init) {
        printf("\n");
    }
    init = 0;

    DIR *dir = opendir (name);
    if (dir == NULL) {
            fprintf(stderr, "%s - Can't access dir \"%s\": %s\n", elf_name, name, strerror(errno));
            return;
        }

    printf("%s:\n", name);

    struct dirent *entry;
    int count = 0;
    while ((entry = readdir(dir)) != NULL) {
        const char *entry_name = entry->d_name;
        if(!all_flag && entry_name[0] == '.')
            continue;
        
        list_node(name, entry_name);
        count++;
    }
    printf("%d files in total.\n", count);
}
\end{lstlisting}

\subsection{递归输出}
有时我们需要实现对于某文件夹的递归输出,此时我们使用广度优先搜索的方式,使用一个链表模拟的队列进行文件夹的递归输出。核心代码如下:

\begin{lstlisting}[language=C]
static void list_main(const char *name) {
    struct stat status;
    if(stat(name, &status)) {
        fprintf(stderr, "%s - Can't access \"%s\": %s\n", elf_name, name, strerror(errno));
        return;
    }
    if(S_ISDIR(status.st_mode)) {
        list_start = malloc(sizeof(list_node_t));
        list_start->node = name;
        list_start->next = NULL;
        list_end = list_start;
        while (list_start) {
            list_dir(list_start->node);
            list_start = list_start->next;
        }
    } else {
        list_node("", name);
    }
}
\end{lstlisting}

\section{运行效果}

\subsection{基本使用}
输出某一文件夹的内容时结果如下:

\begin{lstlisting}
name1e5s@sumeru:~/Homework-2/build$ ./list ..
..:
-rw-rw-r--     1719 2020-04-09 17:31 getopt.c
-rw-rw-r--     5141 2020-04-09 17:31 list.c
-rw-rw-r--      123 2020-04-09 17:31 CMakeLists.txt
-rw-rw-r--      559 2020-04-09 17:31 homemade_getopt.h
5 files in total.
\end{lstlisting}

\subsection{递归输出}
递归输出某一文件夹的内容时结果如下:

\begin{lstlisting}
name1e5s@sumeru:~/Homework-2/build$ ./list ..
..:
-rw-rw-r--     1719 2020-04-09 17:31 getopt.c
-rw-rw-r--     5141 2020-04-09 17:31 list.c
-rw-rw-r--      123 2020-04-09 17:31 CMakeLists.txt
-rw-rw-r--      559 2020-04-09 17:31 homemade_getopt.h
5 files in total.
name1e5s@sumeru:~/Homework-2/build$ ./list -r ..
..:
-rw-rw-r--     1719 2020-04-09 17:31 getopt.c
-rw-rw-r--     5141 2020-04-09 17:31 list.c
-rw-rw-r--      123 2020-04-09 17:31 CMakeLists.txt
-rw-rw-r--      559 2020-04-09 17:31 homemade_getopt.h
drwxrwxr-x     4096 2020-04-09 17:31 build
5 files in total.

../build:
-rw-rw-r--     1502 2020-04-09 17:31 cmake_install.cmake
-rwxrwxr-x    13832 2020-04-09 17:31 list
drwxrwxr-x     4096 2020-04-09 17:31 CMakeFiles
-rw-rw-r--     5304 2020-04-09 17:31 Makefile
-rw-rw-r--    12576 2020-04-09 17:31 CMakeCache.txt
5 files in total.

../build/CMakeFiles:
-rw-rw-r--     3067 2020-04-09 17:31 Makefile2
-rw-rw-r--       85 2020-04-09 17:31 cmake.check_cache
-rwxrwxr-x    12312 2020-04-09 17:31 feature_tests.bin
drwxrwxr-x     4096 2020-04-09 17:31 list.dir
drwxrwxr-x     4096 2020-04-09 17:31 CMakeTmp
-rw-rw-r--      688 2020-04-09 17:31 feature_tests.c
drwxrwxr-x     4096 2020-04-09 17:31 3.10.2
-rw-rw-r--        2 2020-04-09 17:31 progress.marks
-rw-rw-r--    44929 2020-04-09 17:31 CMakeOutput.log
-rw-rw-r--      632 2020-04-09 17:31 CMakeDirectoryInformation.cmake
-rw-rw-r--    10011 2020-04-09 17:31 feature_tests.cxx
-rw-rw-r--     6639 2020-04-09 17:31 Makefile.cmake
-rw-rw-r--      171 2020-04-09 17:31 TargetDirectories.txt
13 files in total.

../build/CMakeFiles/list.dir:
-rw-rw-r--      524 2020-04-09 17:31 C.includecache
-rw-rw-r--      261 2020-04-09 17:31 cmake_clean.cmake
-rw-rw-r--       94 2020-04-09 17:31 link.txt
-rw-rw-r--     3136 2020-04-09 17:31 getopt.c.o
-rw-rw-r--      290 2020-04-09 17:31 depend.make
-rw-rw-r--      320 2020-04-09 17:31 depend.internal
-rw-rw-r--      171 2020-04-09 17:31 flags.make
-rw-rw-r--     7424 2020-04-09 17:31 list.c.o
-rw-rw-r--     5633 2020-04-09 17:31 build.make
-rw-rw-r--       64 2020-04-09 17:31 progress.make
-rw-rw-r--      657 2020-04-09 17:31 DependInfo.cmake
11 files in total.

../build/CMakeFiles/CMakeTmp:
0 files in total.

../build/CMakeFiles/3.10.2:
-rw-r--r--      402 2020-04-09 17:31 CMakeSystem.cmake
-rwxrwxr-x     8248 2020-04-09 17:31 CMakeDetermineCompilerABI_C.bin
-rw-r--r--     4849 2020-04-09 17:31 CMakeCXXCompiler.cmake
drwxrwxr-x     4096 2020-04-09 17:31 CompilerIdC
-rwxrwxr-x     8264 2020-04-09 17:31 CMakeDetermineCompilerABI_CXX.bin
drwxrwxr-x     4096 2020-04-09 17:31 CompilerIdCXX
-rw-r--r--     2219 2020-04-09 17:31 CMakeCCompiler.cmake
7 files in total.

../build/CMakeFiles/3.10.2/CompilerIdC:
-rwxrwxr-x     8408 2020-04-09 17:31 a.out
drwxrwxr-x     4096 2020-04-09 17:31 tmp
-rw-rw-r--    18076 2020-04-09 17:31 CMakeCCompilerId.c
3 files in total.

../build/CMakeFiles/3.10.2/CompilerIdCXX:
-rw-rw-r--    17631 2020-04-09 17:31 CMakeCXXCompilerId.cpp
-rwxrwxr-x     8416 2020-04-09 17:31 a.out
drwxrwxr-x     4096 2020-04-09 17:31 tmp
3 files in total.

../build/CMakeFiles/3.10.2/CompilerIdC/tmp:
0 files in total.

../build/CMakeFiles/3.10.2/CompilerIdCXX/tmp:
0 files in total.
\end{lstlisting}

\subsection{添加限制}
有限制的输出某一文件夹的内容时结果如下:

\begin{lstlisting}
name1e5s@sumeru:~/Homework-2/build$  ./list -l 5000 -m 2 .
.:
-rwxrwxr-x    13832 2020-04-09 17:31 list
-rw-rw-r--     5304 2020-04-09 17:31 Makefile
-rw-rw-r--    12576 2020-04-09 17:31 CMakeCache.txt
5 files in total.

\end{lstlisting}

\appendix
\section{代码清单}

\subsection{CMakeLists.txt}

\begin{lstlisting}
cmake_minimum_required(VERSION 3.0)
project(list)

set(PROJ_FILES list.c getopt.c)

add_executable(list ${PROJ_FILES})
\end{lstlisting}

\subsection{homemade\_getopt.h}

\begin{lstlisting}[language=C]
#ifndef HOMEMADE_GETOPT_H
#define HOMEMADE_GETOPT_H

#ifndef _HOMEMADE_SRC
#include <getopt.h>
#else

#if defined(__cplusplus)
extern "C" {
#endif


#define no_argument 1
#define required_argument 2
#define optional_argument 3

extern char* optarg;
extern int optind;

struct option {
  const char* name;
  int has_arg;
  int* flag;
  int val;
};

int homemade_getopt(int argc, char* const argv[], const char* optstring);
#if defined(__cplusplus)
}
#endif

#define getopt homemade_getopt

#endif

#endif // HOMEMADE_GETOPT_H
\end{lstlisting}

\subsection{getopt.c}

\begin{lstlisting}[language=C]
#include <stddef.h>
#include <stdio.h>
#include <string.h>

#define _HOMEMADE_SRC
#include "homemade_getopt.h"

// Global definitions
char *optarg;
int optind = 1;

static char *optcursor;

int homemade_getopt(int argc, char* const argv[], const char* optstring) {
    int optchar = EOF;

    const char *optdecl = NULL;

    // Initialize global vars
    optarg = NULL;

    // Arguments overflow #1
    if(optind >= argc)
        goto no_more_argument;

    // Arguments overflow #2
    if(argv[optind] == NULL)
        goto no_more_argument;
    
    // Invalid arguments #1
    if(argv[optind][0] != '-')
        goto no_more_argument;
    
    // Invalid arguments #2
    if(strcmp(argv[optind], "-") == 0)
        goto no_more_argument;
    
    // Long arguments #1
    // Skip it here
    if(strcmp(argv[optind], "--") == 0) {
        optind++;
        goto no_more_argument;   
    }

    if (optcursor == NULL || optcursor[0] == '\0')
        optcursor = argv[optind] + 1;

    optchar = optcursor[0];

    optdecl = strchr(optstring, optchar);
    if (optdecl) {
        if (optdecl[1] == ':') {
            optarg = ++optcursor;
            if (*optarg == '\0') {
                if (++optind < argc) {
                    optarg = argv[optind];
                } else {
                    optarg = NULL;
                    optchar = (optstring[0] == ':') ? ':' : '?';
                }
            }
            optcursor = NULL;
        }
    } else {
        optchar = '?';
    }

    if (optcursor == NULL || *++optcursor == '\0')
        ++optind;

    return optchar;
no_more_argument:
    optcursor = NULL;
    return EOF;
}
\end{lstlisting}

\subsection{list.c}

\begin{lstlisting}[language=C]
#define _HOMEMADE_SRC
#include "homemade_getopt.h"

#include <stdio.h> 
#include <stdlib.h>
#include <string.h>
#include <errno.h>
#include <time.h>

#include <dirent.h>
#include <sys/types.h>
#include <sys/stat.h>

// Global options
static const char *elf_name;
static int recursive_flag;
static int all_flag;
static off_t lo_size;
static off_t hi_size;
static time_t modify_time;
static int init = 1;

// Helper linked list
typedef struct lnode {
    const char *node;
    struct lnode *next;
} list_node_t;

typedef list_node_t *list_t;

static list_t list_start;
static list_t list_end;

static char get_type(mode_t mode) {
    if(S_ISREG(mode))
        return '-';
    if(S_ISDIR(mode))
        return 'd';
    if(S_ISCHR(mode))
        return 'c';
    if(S_ISBLK(mode))
        return 'b';
    if(S_ISLNK(mode))
        return 'l';
    if(S_ISFIFO(mode))
        return 'p';
    if(S_ISSOCK(mode))
        return 's';
}

static void list_node(const char *prefix, const char *file_name) {
    size_t name_size = strlen(prefix);
    size_t entry_size = strlen(file_name);
    char *node_name = malloc(name_size + entry_size + 2);

    strcpy(node_name, prefix);
    node_name[name_size] = '/';
    strcpy (&node_name[name_size + 1], file_name);
    node_name[name_size + 1 + entry_size] = 0;

    struct stat status;
    if(stat(node_name, &status)) {
        fprintf(stderr, "%s - Can't access \"%s\": %s\n", elf_name, node_name, strerror(errno));
        return;
    }

    mode_t stat_mode = status.st_mode;
    off_t stat_size = status.st_size;
    time_t stat_mtime = status.st_mtime;

    char mode_text[] = {'-', '-', '-', '-', '-',
                        '-', '-', '-', '-', '-', '\0'};
    mode_text[0] = get_type(stat_mode);
    if(stat_mode & S_IRUSR) {
        mode_text[1] = 'r';
    }
    if(stat_mode & S_IWUSR) {
        mode_text[2] = 'w';
    }
    if(stat_mode & S_IXUSR) {
        mode_text[3] = 'x';
    }
    if(stat_mode & S_IRGRP) {
        mode_text[4] = 'r';
    }
    if(stat_mode & S_IWGRP) {
        mode_text[5] = 'w';
    }
    if(stat_mode & S_IXGRP) {
        mode_text[6] = 'x';
    }
    if(stat_mode & S_IROTH) {
        mode_text[7] = 'r';
    }
    if(stat_mode & S_IWOTH) {
        mode_text[8] = 'w';
    }
    if(stat_mode & S_IXOTH) {
        mode_text[9] = 'x';
    }

    int filter_flag = (!modify_time || time (NULL) - stat_mtime <= modify_time) &&
        (!lo_size || stat_size >= lo_size) &&
        (!hi_size || stat_size <= hi_size);
    if((filter_flag && !S_ISDIR(stat_mode)) || (S_ISDIR(stat_mode) && recursive_flag)) {
        char stat_time_str[64];
        strftime(stat_time_str, 64, 
            "%Y-%m-%d %H:%M", localtime(&stat_mtime));
        printf ("%s %8ld %s %s\n", mode_text, stat_size, stat_time_str, file_name);
    }
    
	int dame_flag = 0;
	size_t file_name_len = strlen(file_name);
	if((file_name[0] == '.' && file_name_len == 1) ||
	   (file_name[0] == '.' && file_name[1] == '.' && file_name_len == 2)) {
		dame_flag = 1;
	}
	
    if(S_ISDIR(stat_mode) && recursive_flag && !dame_flag) {
        list_t tmp = malloc(sizeof(list_node_t));
        tmp->node = node_name;
        tmp->next = NULL;
        list_end->next = tmp;
        list_end = tmp;
    }
}

static void list_dir(const char *name) {
    if(!init) {
        printf("\n");
    }
    init = 0;

    DIR *dir = opendir (name);
    if (dir == NULL) {
            fprintf(stderr, "%s - Can't access dir \"%s\": %s\n", elf_name, name, strerror(errno));
            return;
        }

    printf("%s:\n", name);

    struct dirent *entry;
    int count = 0;
    while ((entry = readdir(dir)) != NULL) {
        const char *entry_name = entry->d_name;
        if(!all_flag && entry_name[0] == '.')
            continue;
        
        list_node(name, entry_name);
        count++;
    }
    printf("%d files in total.\n", count);
}

static void list_main(const char *name) {
    struct stat status;
    if(stat(name, &status)) {
        fprintf(stderr, "%s - Can't access \"%s\": %s\n", elf_name, name, strerror(errno));
        return;
    }
    if(S_ISDIR(status.st_mode)) {
        list_start = malloc(sizeof(list_node_t));
        list_start->node = name;
        list_start->next = NULL;
        list_end = list_start;
        while (list_start) {
            list_dir(list_start->node);
            list_start = list_start->next;
        }
    } else {
        list_node("", name);
    }
}

int main(int argc, char **argv) {
    elf_name = argv[0];
    int c = 0;
    while ((c = getopt (argc, argv, "ral:h:m:")) != -1) {
        switch (c) {
            case 'r':
                recursive_flag = 1;
                break;
            case 'a':
                all_flag = 1;
                break;
            case 'l':
                lo_size = atoi(optarg);
                break;
            case 'h':
                hi_size = atoi(optarg);
                break;
            case 'm':
                modify_time = atoi (optarg) * 24 * 60 * 60;
                break;
            default:
                break;
        }
    }
    for(int i = optind; i < argc; i++) {
        list_main(argv[i]);
    }
    return 0;
}
\end{lstlisting}
\end{document}