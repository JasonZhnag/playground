% Koko
\documentclass[blue,normal,cn]{elegantnote}
\usepackage{array}
\usepackage{courier}
\usepackage{xcolor}
\definecolor{light-gray}{gray}{0.95}
\newcommand{\code}[1]{\colorbox{light-gray}{\texttt{#1}}}
\newfontfamily\courier{Courier New}
\lstset{linewidth=1.1\textwidth,
	numbers=left,
	basicstyle=\small\courier,
	numberstyle=\tiny\courier,
	keywordstyle=\color{blue}\courier,
	commentstyle=\it\color[cmyk]{1,0,1,0}\courier, 
	stringstyle=\it\color[RGB]{128,0,0}\courier,
	frame=single,
	backgroundcolor=\color[RGB]{245,245,244},
	breaklines,
	extendedchars=false, 
	xleftmargin=2em,xrightmargin=2em, aboveskip=1em,
	tabsize=4, 
	showspaces=false
	basicstyle=\small\courier
}
\title{上机作业4:Shell脚本程序模拟}
\version{$\aleph$}
\date{\today}

\begin{document}
\author{
    \begin{tabular}[t]{c}
        088于海鑫  \\
        2017211240 \\
        \emph{name1e5s@bupt.edu.cn}
    \end{tabular}
}
\maketitle

\section{题目要求}

使用 \code{fork()}, \code{exec()}, \code{dup2()}, \code{pipe()} ,\code{open()},\code{wait()} 等系统调用编写C 语言程序完成与下列 Shell 命令等价的功能。

\begin{lstlisting}
grep -v usr < /etc/passwd | wc –l > r.txt; cat r.txt
\end{lstlisting}

(提示:为简化编程,不需要用 \code{strtok} 断词,直接用 \code{execlp} 实现能达到 Shell 命令相同功能的程序即可)

例如:\code{execlp("grep", "grep", "-v", "usr", 0);}

\section{作业答案}

本次作业我们只需使用一些简单的函数调用对 Shell 命令进行模拟即可,主要是模拟管道这一操作。代码如下:

\begin{lstlisting}[language=C]
#include <stdio.h>
#include <stdlib.h>
#include <fcntl.h>
#include <unistd.h>
#include <sys/wait.h>

int main(void) {
    int fd[2];
    pipe(fd);

    if(fork() == 0) {
        int passwd_fd = open("/etc/passwd", O_RDONLY);
        dup2(passwd_fd, 0);
        dup2(fd[1], 1);
        close(fd[0]);
        close(fd[1]);
        execlp("grep", "grep", "-v", "usr", NULL);
    } else {
        if(fork() == 0) {
            int r_fd = open("r.txt", O_WRONLY | O_CREAT, 0777);
            dup2(r_fd, 1);
            dup2(fd[0], 0);
            close(r_fd);
            close(fd[0]);
            close(fd[1]);
            execlp("wc", "wc", "-l", NULL);
        } else {
            wait(NULL);
            execlp("cat", "cat", "r.txt", NULL);
        }
    }
}
\end{lstlisting}

在这里我们使用 \code{pipe} 初始化管道,之后使用 \code{fork} 创建子进程,然后使用 \code{dup2} 替换掉子进程的输入/输出,最后分别调用 \code{execlp} 替换进程映像。这样即可模拟我上面的函数。

测试结果:

\begin{lstlisting}
name1e5s@sumeru:~$ gcc t.c
name1e5s@sumeru:~$ ./a.out
8
name1e5s@sumeru:~$ grep -v usr < /etc/passwd | wc -l > r.txt; cat r.txt
8
\end{lstlisting}

符合要求。

\end{document}