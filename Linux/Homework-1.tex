% Koko
\documentclass[blue,normal,cn]{elegantnote}
\usepackage{array}
\usepackage{courier}
\usepackage{xcolor}
\definecolor{light-gray}{gray}{0.95}
\newcommand{\code}[1]{\colorbox{light-gray}{\texttt{#1}}}
\newfontfamily\courier{Courier New}
\lstset{linewidth=1.1\textwidth,
	numbers=left,
	basicstyle=\small\courier,
	numberstyle=\tiny\courier,
	keywordstyle=\color{blue}\courier,
	commentstyle=\it\color[cmyk]{1,0,1,0}\courier, 
	stringstyle=\it\color[RGB]{128,0,0}\courier,
	frame=single,
	backgroundcolor=\color[RGB]{245,245,244},
	breaklines,
	extendedchars=false, 
	xleftmargin=2em,xrightmargin=2em, aboveskip=1em,
	tabsize=4, 
	showspaces=false
	basicstyle=\small\courier
}
\title{上机作业1:正则表达式应用}
\version{$\aleph$}
\date{\today}

\begin{document}
\author{
\begin{tabular}[t]{c} 
    088于海鑫 \\
    2017211240 \\
    \emph{name1e5s@bupt.edu.cn}
\end{tabular}
}
\maketitle

\section{实验要求}

从因特网上搜索Web页,用wget获取网页,处理网页html文本数
据,从中提取出当前时间点北京各监测站的PM2.5浓度,输出如
下CSV格式数据。
\begin{lstlisting}
2020-03-09 13:00:00,海淀区万柳,73
2020-03-09 13:00:00,昌平镇,67
2020-03-09 13:00:00,奥体中心,66
2020-03-09 13:00:00,海淀区万柳,73
2020-03-09 13:00:00,昌平镇,73
2020-03-09 13:00:00,奥体中心,75
\end{lstlisting}
撰写实验报告。

\section{实验过程}

\subsection{获取数据}

在 \code{bing} 上搜索 ”北京各监测站的PM2.5浓度“ 并筛选,最终我们选定了最为简单的 \code{http://www.86pm25.com/city/beijing.html} 作为筛选的目标。在这个网页中,我们要进行处理的 HTML 片段如下:
\begin{lstlisting}[language=HTML]
     	<!--站点实时值开始-->

<div  style=" margin:auto; width:960px; text-align:center;font-size:14px; font-weight:bold; margin:10px;">各监测站点实时数据</div>
<table >	 <thead onclick="sortColumn(event)"><tr style=" background-color:#EBEFF7"><th width="15%"  >监测站点</th><th width="10%" >AQI</th><th width="20%"  >&nbsp;&nbsp;&nbsp;污染等级</th><th width="20%"  >PM2.5浓度</th> <th width="20%"  >PM10浓度</th> </tr></thead>
<tr><td>奥体中心</td><td>25</td><td><img src="../images/wurandengji/you.gif"  /> </td><td>8μg/m³</td><td>21μg/m³</td></tr>
<tr><td>昌平镇</td><td>33</td><td><img src="../images/wurandengji/you.gif"  /> </td><td>15μg/m³</td><td>33μg/m³</td></tr>
<tr><td>定陵</td><td>26</td><td><img src="../images/wurandengji/you.gif"  /> </td><td>12μg/m³</td><td>24μg/m³</td></tr>
<tr><td>东四</td><td>27</td><td><img src="../images/wurandengji/you.gif"  /> </td><td>14μg/m³</td><td>25μg/m³</td></tr>
<tr><td>古城</td><td>25</td><td><img src="../images/wurandengji/you.gif"  /> </td><td>16μg/m³</td><td>21μg/m³</td></tr>
<tr><td>官园</td><td>30</td><td><img src="../images/wurandengji/you.gif"  /> </td><td>11μg/m³</td><td>20μg/m³</td></tr>
<tr><td>海淀区万柳</td><td>27</td><td><img src="../images/wurandengji/you.gif"  /> </td><td>15μg/m³</td><td>—μg/m³</td></tr>
<tr><td>怀柔镇</td><td>29</td><td><img src="../images/wurandengji/you.gif"  /> </td><td>6μg/m³</td><td>—μg/m³</td></tr>
<tr><td>农展馆</td><td>27</td><td><img src="../images/wurandengji/you.gif"  /> </td><td>16μg/m³</td><td>22μg/m³</td></tr>
<tr><td>顺义新城</td><td>29</td><td><img src="../images/wurandengji/you.gif"  /> </td><td>2μg/m³</td><td>12μg/m³</td></tr>
<tr><td>天坛</td><td>43</td><td><img src="../images/wurandengji/you.gif"  /> </td><td>9μg/m³</td><td>43μg/m³</td></tr>
<tr><td>万寿西宫</td><td>23</td><td><img src="../images/wurandengji/you.gif"  /> </td><td>5μg/m³</td><td>22μg/m³</td></tr></table>
	<!--站点实时值结束-->
\end{lstlisting}

\subsection{数据筛选}

\subsubsection{提取表格}

我们首先要做的就是把这一部分表格从整个 \code{HTML} 文本中筛选出来,经过观察可以发现该网页的表格十分有规律,\textbf{有且只有}表格部分的开头为 \code{<tr><td>} ,以此为切入点,我们使用 \code{grep} 配合正则表达式抓取表格片段。

\begin{lstlisting}
bash$ grep "^<tr><td>"
\end{lstlisting}

\subsection{删除 TAG}

HTML 作为富文本文档,在转换为我们需要的格式前必须要做的就是把多余的 TAG 部分给清除掉,这里我们使用 \code{sed} 来解决问题,代码如下:

\begin{lstlisting}
bash$ sed -E 's/(<[^>]+>)+/~/g' | sed 's/~/ /g'
\end{lstlisting}

这一段代码的作用时先把全部的 TAG 转换为单个的波浪号字符,之后再把该字符转换为空格。此时的输出如下:

\begin{lstlisting}
name1e5s@sumeru:~$ wget -q -O -  http://www.86pm25.com/city/beijing.html | grep "^<tr><td>" | sed -E 's/(<[^>]+>)+/~/g' | sed 's/~/ /g
'
 奥体中心 28   13μg/m³ 21μg/m³
 昌平镇 28   7μg/m³ 10μg/m³
 定陵 29   7μg/m³ 12μg/m³
 东四 30   17μg/m³ 28μg/m³
 古城 29   13μg/m³ 17μg/m³
 官园 33   11μg/m³ 18μg/m³
 海淀区万柳 30   10μg/m³ 10μg/m³
 农展馆 34   17μg/m³ 34μg/m³
 顺义新城 32   22μg/m³ 28μg/m³
 天坛 37   17μg/m³ 37μg/m³
 万寿西宫 24   13μg/m³ 24μg/m³
\end{lstlisting}

\subsection{删除单位}

\code{μg/m³} 作为不需要的单位,使用 \code{sed} 剔除掉。

\begin{lstlisting}
bash$ sed 's%μg/m³%%g'
\end{lstlisting}

此时的输出如下:

\begin{lstlisting}
name1e5s@sumeru:~$ wget -q -O -  http://www.86pm25.com/city/beijing.html | grep "^<tr><td>" | sed -E 's/(<[^>]+>)+/~/g' | sed 's/~/ /g' | sed 's%μg/m³%%g'
 奥体中心 28   13 21
 昌平镇 28   7 10
 定陵 29   7 12
 东四 30   17 28
 古城 29   13 17
 官园 33   11 18
 海淀区万柳 30   10 10
 农展馆 34   17 34
 顺义新城 32   22 28
 天坛 37   17 37
 万寿西宫 24   13 24
\end{lstlisting}

\subsection{打印结果}

在获取到必要的信息之后,我们可以使用 \code{awk} 输出结果。

\begin{lstlisting}
bash$ awk '{print datenow "," $1 "," $3}' datenow="`date "+%Y-%m-%d %H:%M:%S"`"
\end{lstlisting}

此时的输出如下:

\begin{lstlisting}
name1e5s@sumeru:~$ wget -q -O -  http://www.86pm25.com/city/beijing.html | grep "^<tr><td>" | sed -E 's/(<[^>]+>)+/~/g' | sed 's/~/ /g' | sed 's%μg/m³%%g'| awk '{print datenow "," $1 "," $3}' datenow="`date "+%Y-%m-%d %H:%M:%S"`"
2020-03-23 14:35:43,奥体中心,13
2020-03-23 14:35:43,昌平镇,7
2020-03-23 14:35:43,定陵,7
2020-03-23 14:35:43,东四,17
2020-03-23 14:35:43,古城,13
2020-03-23 14:35:43,官园,11
2020-03-23 14:35:43,海淀区万柳,10
2020-03-23 14:35:43,农展馆,17
2020-03-23 14:35:43,顺义新城,22
2020-03-23 14:35:43,天坛,17
2020-03-23 14:35:43,万寿西宫,13
\end{lstlisting}

符合要求。

\end{document}