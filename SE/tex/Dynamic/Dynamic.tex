% Koko
\documentclass[black,normal,cn]{elegantnote}
\usepackage{array}
\usepackage{courier}
\usepackage{xcolor}
\usepackage{tabulary}
\usepackage{float}
\usepackage{makecell}
\usepackage{multirow}
\usepackage{zhnumber}

\definecolor{light-gray}{gray}{0.95}
\newcommand{\code}[1]{\colorbox{light-gray}{\texttt{#1}}}
\newfontfamily\courier{Courier New}
\lstset{linewidth=1.1\textwidth,
	numbers=left,
	basicstyle=\small\courier,
	numberstyle=\tiny\courier,
	keywordstyle=\color{blue}\courier,
	commentstyle=\it\color[cmyk]{1,0,1,0}\courier, 
	stringstyle=\it\color[RGB]{128,0,0}\courier,
	frame=single,
	backgroundcolor=\color[RGB]{245,245,244},
	breaklines,
	extendedchars=false, 
	xleftmargin=2em,xrightmargin=2em, aboveskip=1em,
	tabsize=4, 
	showspaces=false
	basicstyle=\small\courier
}
\title{分布式温控系统用例模型}
\version{1.1.1}
\date{\zhtoday}

\begin{document}
\author{
    \begin{tabular}[t]{c}
        2017211305 班 E 组 \\
        于海鑫\ 徐翔\ 赵泉斌\ 郭朝宇\ 郭璐
    \end{tabular}
}
\maketitle

\tableofcontents

\section{文档介绍}

\subsection{文档说明}

该软件采用控制器模式进行动态结构的设计,根据用例划分各用例控制器。

对于作业要求的“系统的交互图设计部分不要求持久化层的对象设计”,我们理解为无需做持久化层的类的设计,但是持久化过程仍需表现出来。


\subsection{参考文献}

\begin{itemize}
    \item 《\emph{软件工程模型与方法}》 肖丁等\ 北京邮电大学出版社
    \item 《\emph{课程作业\_用例模型\_操作契约 参考答案}》\ 肖丁
\end{itemize}

\subsection{人员分工}

\begin{center}
    \begin{tabular}{cc}
        \toprule
        \textbf{组员} & \textbf{分工}              \\
        \midrule
        于海鑫        & 整合动态结构设计、调度策略 \\
        郭朝宇        & 负责入住客户动态结构设计   \\
        赵泉斌        & 负责前台人员动态结构设计   \\
        郭璐          & 负责管理员动态结构设计     \\
        徐翔          & 负责经理动态结构设计       \\
        \bottomrule
    \end{tabular}
\end{center}

\section{动态结构:入住客户}
\subsection{RequestOn}
\subsubsection{操作契约}

\begin{enumerate}[leftmargin=3em, listparindent=4em, parsep=0pt]
    \item 调度对象与房间建立关联;
    \item 一个服务对象被创建(当前服务对象数小于服务对象数上限,验收环境的服务对象上限数=3);
    \item 调度对象与服务对象建立关联;
    \item 服务对象与房间建立关联;
    \item 调度对象的服务对象数及服务开始时间被赋值;
    \item 服务对象的服务状态,服务开始时间,模式,目标温度,费率及费用值被赋值;
\end{enumerate}

\noindent 当 RequestNumber>ServiceNumber 时:

\begin{enumerate}[leftmargin=3em, listparindent=4em, parsep=0pt]
    \item 调度对象与房间建立关联;
    \item 当前服务对象数大于等于服务对象数上限,则将房间的请求放到等待队列进行调度;
    \item 队列中的等待服务的房间的等待时长被赋值(时间片时长);
    \item 调度对象保存
\end{enumerate}

\subsubsection{设计用例实现过程}

\end{document}