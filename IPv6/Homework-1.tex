% Koko
\documentclass[blue,normal,cn]{elegantnote}
\usepackage{array}
\usepackage{courier}
\usepackage{xcolor}
\definecolor{light-gray}{gray}{0.95}
\newcommand{\code}[1]{\colorbox{light-gray}{\texttt{#1}}}
\newfontfamily\courier{Courier New}
\lstset{linewidth=1.1\textwidth,
	numbers=left,
	basicstyle=\small\courier,
	numberstyle=\tiny\courier,
	keywordstyle=\color{blue}\courier,
	commentstyle=\it\color[cmyk]{1,0,1,0}\courier, 
	stringstyle=\it\color[RGB]{128,0,0}\courier,
	frame=single,
	backgroundcolor=\color[RGB]{245,245,244},
	breaklines,
	extendedchars=false, 
	xleftmargin=2em,xrightmargin=2em, aboveskip=1em,
	tabsize=4, 
	showspaces=false
	basicstyle=\small\courier
}
\title{作业1: 理解自己设备的 IPv6 地址 }
\version{$\aleph$}
\date{\today}

\begin{document}
\author{
	\begin{tabular}[t]{c}
		于海鑫     \\
		2017211240 \\
		\emph{name1e5s@bupt.edu.cn}
	\end{tabular}
}
\maketitle

\section{作业要求}

理解自己的计算机获得的 IPv6 地址。

\begin{enumerate}
	\item 使用手机热点(AP)的方式,将自己的电脑连接到手机的热点(AP)上(在校同学可连接到校园网)
	\item Windows 操作系统的同学,在命令行模式下,用 \code{ipconfig /all} 命令,列出自己电脑所有的网络连接;其他操作系统的同学,在控制台执行相应的命令
	\item 在所有的网络连接里面,找到正在使用的这个连接(连接到手机 AP 的连接,通常是某个 wireless 的网络适配器)
	\item 记录下来获得的 ipv6 地址,对每个 ipv6 地址,分析理解其前缀,并查找自己设备的可汇聚全球单播地址是归属那个运营商的(参考同时发布的图片或到网上查找)
\end{enumerate}

\section{作业内容}

\subsection{获取到的 IPv6 地址}

结果如下:

\begin{lstlisting}
无线局域网适配器 WLAN:

   连接特定的 DNS 后缀 . . . . . . . :
   描述. . . . . . . . . . . . . . . : Realtek RTL8822BE 802.11ac PCIe Adapter
   物理地址. . . . . . . . . . . . . : F8-DA-0C-59-9F-81
   DHCP 已启用 . . . . . . . . . . . : 是
   自动配置已启用. . . . . . . . . . : 是
   IPv6 地址 . . . . . . . . . . . . : 2408:8234:3610:6637:d9d5:92bb:cee9:85dc(首选)
   临时 IPv6 地址. . . . . . . . . . : 2408:8234:3610:6637:cf:b481:c656:7e3(首选)
   本地链接 IPv6 地址. . . . . . . . : fe80::d9d5:92bb:cee9:85dc%8(首选)
   IPv4 地址 . . . . . . . . . . . . : 192.168.1.4(首选)
   子网掩码  . . . . . . . . . . . . : 255.255.255.0
   获得租约的时间  . . . . . . . . . : 2020年3月26日 13:12:23
   租约过期的时间  . . . . . . . . . : 2020年3月27日 13:12:23
   默认网关. . . . . . . . . . . . . : fe80::1%8
                                       192.168.1.1
   DHCP 服务器 . . . . . . . . . . . : 192.168.1.1
   DHCPv6 IAID . . . . . . . . . . . : 100194828
   DHCPv6 客户端 DUID  . . . . . . . : 00-01-00-01-24-8B-BE-B1-3C-52-82-DF-B1-FD
   DNS 服务器  . . . . . . . . . . . : fe80::1%8
                                       192.168.1.1
   TCPIP 上的 NetBIOS  . . . . . . . : 已启用
\end{lstlisting}

\subsection{地址分析}

\subsubsection{2408:8234:3610:6637:d9d5:92bb:cee9:85dc}

\begin{itemize}
	\item \textbf{地址类型}\quad 全局单播
	\item \textbf{前缀}\quad  2408:8234::/32
	\item \textbf{归属}\quad 吉林省联通公众宽带
\end{itemize}

\subsubsection{fe80::d9d5:92bb:cee9:85dc}

\begin{itemize}
	\item \textbf{地址类型}\quad 链路本地地址
	\item \textbf{前缀}\quad  fe80::/10
\end{itemize}

\end{document}